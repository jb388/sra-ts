%% Copernicus Publications Manuscript Preparation Template for LaTeX Submissions
%% ---------------------------------
%% This template should be used for copernicus.cls
%% The class file and some style files are bundled in the Copernicus Latex Package, which can be downloaded from the different journal webpages.
%% For further assistance please contact Copernicus Publications at: production@copernicus.org
%% https://publications.copernicus.org/for_authors/manuscript_preparation.html

%% copernicus_rticles_template (flag for rticles template detection - do not remove!)

%% Please use the following documentclass and journal abbreviations for discussion papers and final revised papers.

%% 2-column papers and discussion papers
\documentclass[soil, manuscript]{copernicus}



%% Journal abbreviations (please use the same for preprints and final revised papers)

% Advances in Geosciences (adgeo)
% Advances in Radio Science (ars)
% Advances in Science and Research (asr)
% Advances in Statistical Climatology, Meteorology and Oceanography (ascmo)
% Annales Geophysicae (angeo)
% Archives Animal Breeding (aab)
% Atmospheric Chemistry and Physics (acp)
% Atmospheric Measurement Techniques (amt)
% Biogeosciences (bg)
% Climate of the Past (cp)
% DEUQUA Special Publications (deuquasp)
% Drinking Water Engineering and Science (dwes)
% Earth Surface Dynamics (esurf)
% Earth System Dynamics (esd)
% Earth System Science Data (essd)
% E&G Quaternary Science Journal (egqsj)
% EGUsphere (egusphere) | This is only for EGUsphere preprints submitted without relation to an EGU journal.
% European Journal of Mineralogy (ejm)
% Fossil Record (fr)
% Geochronology (gchron)
% Geographica Helvetica (gh)
% Geoscience Communication (gc)
% Geoscientific Instrumentation, Methods and Data Systems (gi)
% Geoscientific Model Development (gmd)
% History of Geo- and Space Sciences (hgss)
% Hydrology and Earth System Sciences (hess)
% Journal of Bone and Joint Infection (jbji)
% Journal of Micropalaeontology (jm)
% Journal of Sensors and Sensor Systems (jsss)
% Magnetic Resonance (mr)
% Mechanical Sciences (ms)
% Natural Hazards and Earth System Sciences (nhess)
% Nonlinear Processes in Geophysics (npg)
% Ocean Science (os)
% Polarforschung - Journal of the German Society for Polar Research (polf)
% Primate Biology (pb)
% Proceedings of the International Association of Hydrological Sciences (piahs)
% Safety of Nuclear Waste Disposal (sand)
% Scientific Drilling (sd)
% SOIL (soil)
% Solid Earth (se)
% The Cryosphere (tc)
% Weather and Climate Dynamics (wcd)
% Web Ecology (we)
% Wind Energy Science (wes)

% Pandoc citation processing

% The "Technical instructions for LaTex" by Copernicus require _not_ to insert any additional packages.
% 
% tightlist command for lists without linebreak
\providecommand{\tightlist}{%
  \setlength{\itemsep}{0pt}\setlength{\parskip}{0pt}}


%
\begin{document}


\title{Soil minerals mediate climatic control of soil C cycling on annual to
centennial timescales}


\Author[1]{Jeffrey}{Beem-Miller}
\Author[2]{Craig}{Rasmussen}
\Author[3]{Alison M.}{Hoyt}
\Author[1]{Marion}{Schrumpf}
\Author[4]{Georg}{Guggenberger}
\Author[1]{Susan}{Trumbore}


\affil[1]{Department of Biogeochemical Processes, Max Planck Institute for
Biogeochemistry, Jena, Germany}
\affil[2]{Department of Environmental Science, The University of Arizona, Tucson,
AZ, USA}
\affil[3]{Department of Earth System Science Science, Stanford University,
Stanford, CA, USA}
\affil[4]{Institute of Soil Science, Leibniz University Hannover, Hannover,
Germany}

\runningtitle{Soil minerals mediate climatic control of soil C}

\runningauthor{Beem-Miller et al.}


\correspondence{Jeffrey\ Beem-Miller\ (jbeem@bgc-jena.mpg.de)}



\received{}
\pubdiscuss{} %% only important for two-stage journals
\revised{}
\accepted{}
\published{}

%% These dates will be inserted by Copernicus Publications during the typesetting process.


\firstpage{1}

\maketitle


\begin{abstract}
Climate and parent material both affect soil C persistence, yet the
relative importance of climatic versus mineralogical controls on soil C
dynamics remains unclear. To test this we collected soil samples in
2001, 2009, and 2019 along a combined gradient of parent material
(andesite, basalt, granite) and climate (MAT: 6.5 °C ``cold'', 8.6 °C
``cool'', 12.0 °C ``warm''). We measured the radiocarbon of
heterotrophically respired CO\textsubscript{2}
(\(\Delta\)\textsuperscript{14}C\textsubscript{\emph{respired}}) and
bulk soil C (\(\Delta\)\textsuperscript{14}C\textsubscript{\emph{bulk}})
as proxies for fast and slowly cycling soil C, and characterized mineral
assemblages using selective dissolution. Using linear regression, we
observed that MAT was not a significant predictor of either
\(\Delta\)\textsuperscript{14}C\textsubscript{\emph{bulk}} or
\(\Delta\)\textsuperscript{14}C\textsubscript{\emph{respired}}, yet
climate was highly significant as a categorical variable. Climate
explained more variance in
\(\Delta\)\textsuperscript{14}C\textsubscript{\emph{bulk}} and
\(\Delta\)\textsuperscript{14}C\textsubscript{\emph{respired}} over
0-0.1 m, but parent material explained more from 0.1-0.3 m. Soil C
tended to be older in cool than cold or warm climate soils, and
decreased in age from andesitic to basaltic to granitic soils. Poorly
crystalline metal oxides (PCMs) (but not crystalline metal oxides) were
significantly (p \textless{} 0.1) correlated with
\(\Delta\)\textsuperscript{14}C\textsubscript{\emph{bulk}},
\(\Delta\)\textsuperscript{14}C\textsubscript{\emph{respired}}, and
\(\Delta\)\textsuperscript{14}C\textsubscript{\emph{respired}}-\(\Delta\)\textsuperscript{14}C\textsubscript{\emph{bulk}},
indicating their importance for soil C persistence on both short term
and longer time scales. Change in over time
\(\Delta\)\textsuperscript{14}C\textsubscript{\emph{respired}} was
linearly related to MAT for the granite soils with the lowest PCM
content, but this temperature relationship was absent in the andesitic
and basaltic soils with higher PCM content. We interpret this as
evidence that PCMs may attenuate the temperature sensitivity of
decomposition.
\end{abstract}


\copyrightstatement{The author's copyright for this publication is transferred to
institution/company.}


\introduction[Introduction]

Understanding the response of soil carbon stocks to current and future
changes in climate requires insight into the environmental factors
governing soil carbon dynamics. Climate, and in particular temperature,
has been found to be the most important variable for explaining the age
of soil carbon in topsoil at local to global scales
\citep{frank2012, mathieu2015, shi2020}. Yet our current understanding
of soil organic matter decomposition underscores the importance of
mechanisms that can affect the temperature sensitivity of decomposition,
such as the interaction between soil organic matter and minerals
\citep{davidson2000, davidson2006, rasmussen2005, lehmann2015}. The
effect of mineral-organic associations on the temperature sensitivity of
soil organic matter has been addressed in several modeling studies
\citep{ahrens2020, woolf2019, abramoff2019, tang2014}. These models
typically invoke Michaelis Menten kinetics in addition to an
Arrhenius-type temperature response in order to account for energy and
substrate limitations on decomposition rates for soil organic matter
found in association with minerals \citep{tang2019, ahrens2020}. Studies
comparing the relative importance of different soil mineral assemblages
versus temperature for explaining soil C dynamics over time in situ are
scarce, yet critical for testing model-based findings. We designed the
current study to quantify the relative importance of climatic versus
mineralogical mechanisms of soil organic C persistence across a combined
gradient of mean annual temperature (MAT) and parent material in order
to provide insight into the relevant time scales associated with these
mechanisms.

The relevance of soil minerals for mediating soil organic matter
protection has been found to be a function of the specific minerals
present, rather than the amount of clay or total mineral surface area
\citep{rasmussen2018a, kramer2018}. Soil mineral assemblages are
dynamic, developing over time as primary minerals inherited from parent
material weather to form reactive, poorly crystalline secondary
minerals, which in turn eventually weather to increasing stable
crystalline species \citep{slessarev2022, mikutta2010}. Soils enriched
in poorly crystalline metal oxides (PCM), such as Al and Fe
oxyhydroxides, are known to be of particular importance for the
accumulation and persistence of soil C \citep{torn1997, masiello2004}.
The amount of these minerals present is directly related to parent
material, but is also a function of primary mineral weathering rates
\citep{slessarev2022}. Due to the strong effect of climate on weathering
rates, different soil mineral assemblages can form from the same parent
material under different climatic regimes
\citep{rasmussen2018b, kramer2016}. Conversely, similar mineral
assemblages can be found among soils developed on different parent
materials given adequate time for weathering and similar vegetation and
climate \citep{graham2010}. These complex interactions demonstrate that
climatic and mineralogical controls on soil C cycling are not
independent, but interact over the centennial to millenial time scales
of soil development.

The strength and sorptive capacity of soil minerals is dependent on
ligand exchange, which is a function of not only surface area and
charge, but more specifically the density of accessible hydroxyl groups
\citep{kaiser2003, rasmussen2018a, kleber2015}. Poorly crystalline metal
oxides are particularly enriched in hydroxyl groups, and batch
sorption/desorption experiments have shown that the mineral-organic
interactions between pedogenic metal oxide-rich clays are stronger than
those with siloxane-rich phyllosilicate clays \citep{kahle2004}.
Furthermore, the reactive properties of pedogenic metal oxides can also
facilitate lower strength interactions with soil organic matter through
multivalent cation bridging \citep{kleber2007}. This high reactivity of
poorly crystalline Fe oxides is also implicated in the observations that
poorly crystalline Fe is correlated with aggregate stability but
crystalline Fe oxide content is not \citep{duiker2003}. Consequently,
Rasmussen et al. (2018b) observed that oxalate extractable iron, as an
extractant for PCM, was the best predictor both soil C concentration and
bulk soil radiocarbon
(\(\Delta\)\textsuperscript{14}C\textsubscript{\emph{bulk}}). However,
the relevance of mineral-organic associations with specific mineral
phases such as PCM or crystalline metal oxides (CRM) for specific
timescales of soil C turnover is poorly studied.

Radiocarbon (\textsuperscript{14}C) is a useful tracer for soil C
dynamics over annual to millennial time scales \citep{trumbore2000}. The
use of \textsuperscript{14}C to measure timescales of soil carbon
decomposition is reliant on our knowledge of the ratio of
\textsuperscript{12}C/\textsuperscript{14}C in the atmosphere. Once
CO\textsubscript{2} is fixed into organic matter via photosynthesis,
this ratio starts to shift as \textsuperscript{14}C is preferentially
lost due to radioactive decay. Changes in the
\textsuperscript{12}C/\textsuperscript{14}C ratio due to radioactive
decay are detectable at the relatively longer timescales of hundreds to
thousands of years. However, we can detect changes in
\textsuperscript{14}C with nearly annual resolution for the so-called
``bomb-C'' period, which began with the atmospheric testing of nuclear
weapons in the mid-20\textsuperscript{th} century. This pulse of
``bomb-C'' led to a doubling of atmospheric \textsuperscript{14}C
concentration prior to the ban on above-ground nuclear tests in 1963
\citep{hua2021}. The level of \textsuperscript{14}C in the atmosphere
returned to pre-bomb levels around 2020, thus archived samples now
represent the best opportunity to construct a high-resolution time
series of the bomb-C pulse as it moves through different soil organic
matter pools \citep{trumbore2009}.

Soil is an open system, and this has important implications for the
interpretation of radiocarbon measurements of soil C. For most soils,
the majority of carbon that enters the soil leaves relatively quickly,
with only a small fraction persisting \citep{sierra2018}. Radiocarbon
measurements of bulk soil C
(\(\Delta\)\textsuperscript{14}C\textsubscript{\emph{bulk}}) typically
capture the signal from these persistent pools, while more transient
pools dominate measurements of C leaving the soil via heterotrophic
respiration \citep{trumbore2000}. Here we define ``transient'' for C
cycling on annual to decadal timescales, while we use ``persistent'' to
refer to C cycling on centennial to millennial timescales. A diagnostic
feature of the radiocarbon measurements of
\(\Delta\)\textsuperscript{14}C\textsubscript{\emph{bulk}} and
heterotrophically respired CO\textsubscript{2}
(\(\Delta\)\textsuperscript{14}C\textsubscript{\emph{respired}}) is that
when these two signals are the same it indicates that all of the C in
the soil has an equal probability of being decomposed by microbes and
the system is homogenous (Sierra et al., 2017). However, when
\(\Delta\)\textsuperscript{14}C\textsubscript{\emph{bulk}} and
\(\Delta\)\textsuperscript{14}C\textsubscript{\emph{respired}} are
substantially different this indicates the presence of both labile and
persistent pools of soil C \citep{ewing2006, hopkins2012}. We turned to
the western slope of the Sierra Nevada Mountains, USA to compare and
contrast the effects of climate and mineral assemblage on soil C
dynamics. Drawing on earlier studies in this region
\citep{jenny1949, dahlgren1997, rasmussen2004, trumbore1996}, we
selected soils similar in age and vegetation along a combined gradient
of parent material (granite, andesite, basalt) and MAT (6.5 °C, 8.6 °C,
12.0 °C). The climate gradient also represents a weathering gradient,
with poorly developed soils at the cold climate sites, intermediately
developed soils at the cool climate sites, and highly weathered soils at
the warm climate sites \citep{harradine1958, rasmussen2010a}. Previous
work at these sites \citep{rasmussen2004, rasmussen2018b} (and at nearby
locations
\citetext{\citealp[\citet{koarashi2012}]{trumbore1996}; \citealp{dahlgren1997}; \citealp{castanha2008}})
confirm strong differences in mineral assemblages along both the parent
material and climate gradients, making for an ideal setting to probe the
relative influence of climatic and mineralogical factors and their
interaction on soil C dynamics.

We were able to construct a time series of both
\(\Delta\)\textsuperscript{14}C\textsubscript{\emph{bulk}} and
\(\Delta\)\textsuperscript{14}C\textsubscript{\emph{respired}} at these
sites by combining data from samples newly collected in 2019 with data
from archived samples collected in 2001 and 2009. Such a time series
provides a crucial constraint for determining the trajectory of
bomb-derived \textsuperscript{14}C concentrations over time
\citep{beem-miller2021, stoner2021}. Whether bomb-C concentrations are
increasing or decreasing in bulk or respired CO\textsubscript{2} over
time depends on both on the distribution of soil C among pools with
different cycling rates as well as the year in which the soil was
sampled. Given this, the trajectory of \textsuperscript{14}C cannot be
easily determined from observations at a single point in time
\citep{baisden2013}. Using the radiocarbon time series in combination
with previously determined mineralogical data, we were able to test
several hypotheses regarding the roles of mineralogical versus climatic
factors in determining both overall cycling rates and the dynamics of
transiently cycling soil C.

We can expect soils with large stocks of persistent soil C to have
depleted values of
\(\Delta\)\textsuperscript{14}C\textsubscript{\emph{bulk}} relative to
soils dominated by fast cycling soil C. If climate proves more important
than parent material for determining soil C persistence, than we would
expect to see large differences in
\(\Delta\)\textsuperscript{14}C\textsubscript{\emph{bulk}} among climate
regimes when comparing soils within a given parent material, but minimal
differences among parent materials when comparing soils within the same
climate regime. However, if parent material proves more important than
climate for soil C persistence, we would expect the opposite trends in
\(\Delta\)\textsuperscript{14}C\textsubscript{\emph{bulk}}: differences
would be greater among parent materials within a given climate regime
than among climate regimes within a given parent material.
Alternatively, if persistent soil C is associated with specific soil
minerals, we would expect an interactive effect of parent material and
climate on \(\Delta\)\textsuperscript{14}C\textsubscript{\emph{bulk}}.
For example, if soil C persistence is due to the association of soil
organic matter with poorly crystalline metal oxides, we would expect to
observe the most depleted
\(\Delta\)\textsuperscript{14}C\textsubscript{\emph{bulk}} values where
the combination of parent material and climate factors has led to the
greatest abundance of these specific soil minerals.

Soil C found in association with minerals is typically older than
organic matter found in free particulate forms \citep{lavallee2020}.
Accordingly, we might expect climate to be the dominant factor
controlling the amount and cycling rates of C in transiently cycling
soil C pools, with mineral factors being less relevant at these shorter
timescales. If this hypothesis is correct, we would expect to see
greater differences in
\(\Delta\)\textsuperscript{14}C\textsubscript{\emph{respired}} among
different climate regimes and within a given parent material than we
would among different parent materials within the same climate regime.
Furthermore, given sufficient moisture, we would expect warmer climate
soils to have
\(\Delta\)\textsuperscript{14}C\textsubscript{\emph{respired}} values
closer to the atmosphere than colder climate soils, due to faster
decomposition rates in the actively cycling soil C pools, and
accordingly, we would expect
\(\Delta\)\textsuperscript{14}C\textsubscript{\emph{respired}} to change
more over time at the warmer climate sites than at the colder climate
sites.

\section{Methods}

\subsection{Site descriptions}

We collected samples from nine sites across a combined gradient of
parent material and climate in the Sierra Nevada Mountains of California
(Table 1). Parent material changes from basalt to andesite to granite
along the north-south axis of the cordillera, while MAT decreases as a
function of increasing elevation along the east-west axis. Total mean
annual precipitation (MAP) ranges from 910 to 1400 mm yr-1 across the
sites. Precipitation increases slightly with elevation (Table 1), and
falls mainly as rain at lower elevations (\textless{} 1400 m), but
mainly as snow at higher elevations (\textgreater{} 1800 m). The
andesitic and basaltic parent materials receive slightly more
precipitation on average than the granitic soils, with MAP values of
1330 (± 75) mm yr-1, 1160 (± 175) mm yr-1, and 1000 (± 85) mm yr-1
averaged across the andesite, basalt, and granite transects,
respectively.

Vegetation at the study sites is typical of the Sierra Mixed Conifer
habitat (Parker and Matyas, 1981). All of the sites are forested and
dominated by conifers. The species composition changes along the
elevation and climate gradient, but not along the parent material
gradient. Tree species at the lowest elevation warm climate sites are
predominantly \emph{Pinus ponderosa} mixed with lesser amounts of
\emph{Quercus} spp. The canopy species at the mid-elevation cool climate
sites consist primarily of \emph{Abies concolor} and \emph{Pinus
lambertiana}, while \emph{Abies magnifica} dominates at the high
elevation cold climate sites. Species present at all sites include
Calocedrus decurrens in the canopy, the shrubs \emph{Arctostaphylos}
spp., \emph{Chamaebatia foliolosa}, and \emph{Ceanothus} spp. in the
understory, and patchy ground cover of grasses and forbs (Table 1).

\subsection{Sample collection}

Site locations were initially established in 2001 by C. Rasmussen
\citep{rasmussen2004} and resampled in 2009 \citep{rasmussen2018b} and
2019 (this study). Three replicate pits were dug at each site, with
samples collected from the A horizon in 2001, and from both the A and B
horizons in 2009 and 2019. Samples were collected from the pit sidewalls
by horizon in 2001 and 2009, and by 0.1 m increments in 2019. For the
2019 resampling, we located the sites using GPS and geospatial
coordinates recorded during site establishment. Prior to sample
collection we compared the soil profiles to the pedon descriptions from
the 2001 sampling campaign to confirm the profiles matched. However, we
only focus on the upper mineral soil layers in this study (0 to ca.
0.3m) as the 2001 sampling was restricted to this depth range.

\subsection{Incubations}

Laboratory soil incubations were performed on composite samples from the
three replicate profiles sampled at each site in 2001 and 2019. We
omitted the 2009 samples from the incubation experiment to save on time
and analysis costs, and because sample material was only available from
a single profile at each site. We composited and incubated each depth
increment separately in 1 L glass mason jars fitted with airtight
sampling ports in the lids. Incubations were performed in duplicate.
Prior to the start of incubations, we adjusted the soil moisture content
to 60\% of water holding capacity (WHC). Samples from 2001 were
air-dried prior to archiving, and therefore we also air-dried the
freshly collected soils from 2019 in order to control for the known
effects of drying and rewetting on
\(\Delta\)\textsuperscript{14}C\textsubscript{\emph{respired}}
\citep{beem-miller2021}. We defined WHC as the gravimetric water content
of water-saturated soil placed in mesh-covered (50µm) tubes (50ml)
weighed after draining for 30 minutes on a bed of fine sand. Following
rewetting we allowed the soils to respire for one week before closing
the jars. Incubations proceeded until CO\textsubscript{2} concentrations
in the jar headspace reached approximately 10,000 ppm, at which point we
collected a 400 ml gas subsample for radiocarbon analysis. Gas samples
were collected with pre-evacuated stainless-steel vacuum canisters
(Restek GmbH, Bad Homburg, Germany). All incubations were performed in
the dark at 20 °C.

\subsection{Soil Physical Analyses and Mineral Characterization}

Data on soil particle size distributions, bulk density, and mineral
characterization were obtained from previously published analyses of
samples collected at the study sites in 2001 and 2009 (Rasmussen et al.,
2010a, 2018b, 2005, 2010b, 2007). Both qualitative and quantitative
approaches were used to characterize soil mineral assemblages, including
X-ray diffraction (XRD) for the clay (\textless{}2 µm) fraction and
non-sequential selective dissolution. These previous analyses revealed
that the dominant mineral species in the soils of the highly weathered
warm climate zone were similar across parent materials, but differed
substantially across parent materials at the less weathered cool and
cold climate sites. Mineral assemblages at the warm climate sites are
dominated by 1:1 clays and large accumulations of crystalline iron
oxides \citep{rasmussen2010b, dahlgren1997, rasmussen2010a}. In
contrast, the cool and cold climate andesitic soils contain high
concentrations of poorly crystalline short-range order (SRO) minerals
such as allophane and iron oxyhydroxides. The cool and cold climate
basaltic soils contain intermediate amounts of SRO minerals, while the
granitic soils lack SRO minerals almost entirely, but are rich in quartz
and contain relatively more hydroxyl-interlayered vermiculite than soils
from the other lithologies.

The previous work at these sites showed that the PCM content was the
best predictor of both C abundance and
\(\Delta\)\textsuperscript{14}C\textsubscript{\emph{bulk}}
\citep{rasmussen2018b}. Accordingly, our analyses focus on the
relationship between radiocarbon measurements and the abundance of
poorly crystalline versus crystalline metal oxides as well as 2:1 versus
1:1 clays, but not the whole suite of mineralogical data. For
simplification, we use the sum of ammonium-oxalate extractable aluminum
and half of the ammonium-oxalate extractable Fe selectively dissolved
from bulk soils as a proxy for the abundance of poorly and
non-crystalline metal oxides, and the difference of dithionite-citrate
extractable Fe and ammonium-oxalate extractable Fe for the abundance of
crystalline Fe \citep{kleber2005}. We present the results of regression
analyses looking at the specific relationships between
\(\Delta\)\textsuperscript{14}C\textsubscript{\emph{bulk}},
\(\Delta\)\textsuperscript{14}C\textsubscript{\emph{respired}}, and the
concentration of Fe or Al extracted with ammonium-oxalate, Fe extracted
with dithionite-citrate, and Al extracted with sodium-pyrophosphate in
the supplemental information.

\subsection{Carbon, Nitrogen, and Radiocarbon Analysis}

Total carbon and nitrogen content was determined by dry combustion (2019
samples: Vario Max, Elementar Analysensysteme GmbH, Langenselbold,
Germany) on finely ground soils (2019 samples: MM400, Retsch GmbH, Haan,
Germany). For radiocarbon analysis of 2001 and 2019 samples, we first
purified CO\textsubscript{2} from combusted soil samples (bulk soils)
and incubation flask samples (respired CO\textsubscript{2}) on a vacuum
line using liquid N\textsubscript{2}. Following purification, samples
were graphitized with an iron catalyst under an H2 enriched atmosphere
at 550 °C. Radiocarbon content was then measured by accelerator mass
spectrometry (Micadas, Ionplus, Zurich, Switzerland) at the Max Planck
Institute for Biogeochemisty \citep{steinhof2017}. See Rasmussen et al.
(2018b) for details of C, N, and radiocarbon analysis of the 2009
samples.

We report radiocarbon values using units of
\(\Delta\)\textsuperscript{14}C, defined as the deviation in parts per
thousand of the ratio of \textsuperscript{14}C/\textsuperscript{12}C
from that of the oxalic acid standard measured in 1950. This unit also
contains a correction for the potential effect of mass-dependent
fractionation by normalizing sample \(\delta\)\textsuperscript{13}C to a
common \(\delta\)\textsuperscript{13}C value of -25 per mil
\citep{stuiver1977}. Values with \(\Delta\)\textsuperscript{14}C
\textgreater{}0 indicate the presence of `bomb' C produced by
atmospheric weapons testing in the early 1960s. Values with
\(\Delta\)\textsuperscript{14}C \textless{} 0 indicate the influence of
radioactive decay of \textsuperscript{14}C, which has a half-life of
5730 y.

\subsection{Spline fitting}

We used a spline function to compare soil properties from samples
collected from different depth intervals in different years and at
different sites. We were motivated to use consistent depth increments
across sites when resampling in 2019 because of the strong correlation
between depth and \(\Delta\)\textsuperscript{14}C observed in the 2009
dataset, a correlation also noted in numerous other studies
\citep{mathieu2015, shi2020}. We fit a mass-preserving quadratic spline
to the 2001 and 2009 profiles in order to convert soil property data to
the equivalent depth increments sampled in 2019 \citep{bishop1999}. We
performed the spline fitting with the mpspline function of the GSIF
package in R, using a \(\lambda\) value of 0.1 \citep{hengl2019}.

\subsection{Statistical analysis}

We used a linear modeling approach to assess the relative explanatory
power of climate versus parent material on the observed variation in
\(\Delta\)\textsuperscript{14}C, as well as potential interactions
between these two factors. We constructed separate models for
\(\Delta\)\textsuperscript{14}C\textsubscript{\emph{bulk}} and
\(\Delta\)\textsuperscript{14}C\textsubscript{\emph{respired}} but with
the same equation structure Eq. (1). For each model we considered the
two-way interaction between parent material and climate as well as the
three-way interaction with time Eq. (1). For ease of interpretation, we
considered the effect of depth by modeling each depth layer separately
(0-0.1 m, 0.1-0.2 m, 0.2-0.3 m). We also made pairwise comparisons of
\(\Delta\)\textsuperscript{14}C\textsubscript{\emph{bulk}} and
\(\Delta\)\textsuperscript{14}C\textsubscript{\emph{respired}} across
sites and within years, and across years for individual sites. We
assessed the significance of the temporal trend for pairwise
combinations of parent material and climate using the emmtrends function
of the emmeans package \citep{lenth2021}. We corrected for multiple
comparisons using Tukey's honestly significant mean difference.

\begin{equation}
\Delta^{14}C = \alpha + \beta_{1}(Parent\_material) \times \beta_{2}(Climate) \times \beta_{3}(Year) + \varepsilon,
\end{equation}

where \(\alpha\) is the intercept term, the \(\beta\) terms are
coefficients, and \(\varepsilon\) is random error.

We also considered the relationship between
\(\Delta\)\textsuperscript{14}C\textsubscript{\emph{bulk}} and
\(\Delta\)\textsuperscript{14}C\textsubscript{\emph{respired}} to gain
insight into potential differences in soil C dynamics and persistence
mechanisms across our sites (Sierra et al.~2018). We modeled the effects
of parent material Eq. (2) and climate Eq. (3) on this relationship
separately, as we did not have an adequate number of observations to
consider the interaction between these two explanatory variables. For
this analysis we used \(\Delta\)\textsuperscript{14}C measurements made
on samples collected in 2001 and 2019, and data from all depths. The
three-way interactions of
\(\Delta\)\textsuperscript{14}C\textsubscript{\emph{bulk}} and the
explanatory variables (parent material or climate) were not significant
with either depth or time for either Eq. (2) or Eq. (3), so we did not
include either depth or time as variables in the models.

\begin{equation}
\Delta^{14}C_{respired} = \alpha + \beta_{1}(\Delta^{14}C_{bulk}) \times \beta_{2}(Parent\_material) + \varepsilon
\end{equation}

\begin{equation}
\Delta^{14}C_{respired} = \alpha + \beta_{1}(\Delta^{14}C_{bulk}) \times \beta_{2}(Climate) + \varepsilon
\end{equation}

We assessed the relative importance of poorly crystalline versus
crystalline iron minerals in protecting soil C from microbial
decomposition by regressing \(\Delta\)\textsuperscript{14}C against the
concentrations of ammonium-oxalate extractable iron, ammonium-oxalate
extractable aluminum, pyrophosphate extractable aluminum, and
dithionite-citrate extractable iron Eq. (4). We fit the model for
\(\Delta\)\textsuperscript{14}C\textsubscript{\emph{bulk}},
\(\Delta\)\textsuperscript{14}C\textsubscript{\emph{respired}}, and the
difference between
\(\Delta\)\textsuperscript{14}C\textsubscript{\emph{respired}} and
\(\Delta\)\textsuperscript{14}C\textsubscript{\emph{bulk}}
(\(\Delta\)\textsuperscript{14}C\textsubscript{\emph{bulk}}-respired).
We used \(\Delta\)\textsuperscript{14}C data from 2001, 2009, and 2019
for the \(\Delta\)\textsuperscript{14}C\textsubscript{\emph{bulk}}
model, but only data from 2001 and 2019 for the
\(\Delta\)\textsuperscript{14}C\textsubscript{\emph{respired}} and
\(\Delta\)\textsuperscript{14}C\textsubscript{\emph{bulk}}-respired
models. Selective dissolution was only performed on the soils collected
in 2001, but these data were assumed to be comparable for the other time
points as they reflect weathering processes operating at timescales much
beyond the 18-year duration of this study. The regression analysis
conducted with Eq. (4) was done for the combined depth increment of 0 to
0.3 m, as extracted metal concentrations did not change substantially
over this increment. Combining depth increments allowed us to control
for the depth dependence of \(\Delta\)\textsuperscript{14}C as well as
simplify interpretation of the data. In order to obtain values for the
necessary data over the 0 to 0.3 m depth increment we computed
mass-weighted estimates of extractable metal concentrations, carbon
mass-weighted means of
\(\Delta\)\textsuperscript{14}C\textsubscript{\emph{bulk}} and
flux-weighted means of
\(\Delta\)\textsuperscript{14}C\textsubscript{\emph{respired}}; these
calculations were made prior to determining
\(\Delta\)\textsuperscript{14}C\textsubscript{\emph{bulk-respired}}.

\begin{equation}
\Delta^{14}C = \alpha + \beta_{1}(Metal_{x}) + \beta_{2}(time) + \varepsilon,
\end{equation}

where \(\alpha\) is the intercept term, the \(\beta\) terms are
coefficients, \emph{Metal\textsubscript{x}} is the concentration of
selectively dissolved metal oxides, time is the year of sampling, and
and \(\varepsilon\) is random error.



\codedataavailability{use this to add a statement when having data sets and software code
available} %% use this section when having data sets and software code available



%%%%%%%%%%%%%%%%%%%%%%%%%%%%%%%%%%%%%%%%%%
%% optional

%%%%%%%%%%%%%%%%%%%%%%%%%%%%%%%%%%%%%%%%%%
\appendix
\section{Figures and tables in appendices}

Regarding figures and tables in appendices, the following two options
are possible depending on your general handling of figures and tables in
the manuscript environment:

\subsection{Option 1}

If you sorted all figures and tables into the sections of the text,
please also sort the appendix figures and appendix tables into the
respective appendix sections. They will be correctly named
automatically.

\subsection{Option 2}

If you put all figures after the reference list, please insert appendix
tables and figures after the normal tables and figures.

To rename them correctly to A1, A2, etc., please add the following
commands in front of them: \texttt{\textbackslash{}appendixfigures}
needs to be added in front of appendix figures
\texttt{\textbackslash{}appendixtables} needs to be added in front of
appendix tables

Please add \texttt{\textbackslash{}clearpage} between each table and/or
figure. Further guidelines on figures and tables can be found below.
\noappendix

%%%%%%%%%%%%%%%%%%%%%%%%%%%%%%%%%%%%%%%%%%

%%%%%%%%%%%%%%%%%%%%%%%%%%%%%%%%%%%%%%%%%%
\competinginterests{The authors declare no competing interests.} %% this section is mandatory even if you declare that no competing interests are present

%%%%%%%%%%%%%%%%%%%%%%%%%%%%%%%%%%%%%%%%%%

%%%%%%%%%%%%%%%%%%%%%%%%%%%%%%%%%%%%%%%%%%
\begin{acknowledgements}
Thanks to the rticles contributors.
\end{acknowledgements}

%% REFERENCES
%% DN: pre-configured to BibTeX for rticles

%% The reference list is compiled as follows:
%%
%% \begin{thebibliography}{}
%%
%% \bibitem[AUTHOR(YEAR)]{LABEL1}
%% REFERENCE 1
%%
%% \bibitem[AUTHOR(YEAR)]{LABEL2}
%% REFERENCE 2
%%
%% \end{thebibliography}

%% Since the Copernicus LaTeX package includes the BibTeX style file copernicus.bst,
%% authors experienced with BibTeX only have to include the following two lines:
%%
\bibliographystyle{copernicus}
\bibliography{lib.bib}
%%
%% URLs and DOIs can be entered in your BibTeX file as:
%%
%% URL = {http://www.xyz.org/~jones/idx_g.htm}
%% DOI = {10.5194/xyz}


%% LITERATURE CITATIONS
%%
%% command                        & example result
%% \citet{jones90}|               & Jones et al. (1990)
%% \citep{jones90}|               & (Jones et al., 1990)
%% \citep{jones90,jones93}|       & (Jones et al., 1990, 1993)
%% \citep[p.~32]{jones90}|        & (Jones et al., 1990, p.~32)
%% \citep[e.g.,][]{jones90}|      & (e.g., Jones et al., 1990)
%% \citep[e.g.,][p.~32]{jones90}| & (e.g., Jones et al., 1990, p.~32)
%% \citeauthor{jones90}|          & Jones et al.
%% \citeyear{jones90}|            & 1990


\end{document}
